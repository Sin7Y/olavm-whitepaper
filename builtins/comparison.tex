\subsection{Comparison}

Range Check is used to perform \verb|GTE| comparison, using Cairo's scheme \cite{cryptoeprint:2021/1063}.

On a circuit level, we only need to support 3 comparisons: \verb|EQ|, \verb|GTE| and \verb|NEQ|. We will perform 256-bit Range Check for two numbers to be compared, and then perform a 256-bit Range Check for the difference between two numbers.

\verb|EQ| checks the equality of two 256-bit numbers $a$ and $b$. The constraint is $a-b=0$.

\verb|NEQ| checks the inequality of two 256-bit numbers $a$ and $b$. The constraint is $(a-b)\cdot(a-b)^{-1}=1$.

\verb|GTE| checks whether two 256-bit numbers $a$ and $b$ satisfy $a \geq b$.

First, use Range Check to constrain each number in 256 bits, and then compare them according to the difference. If the difference exceeds 256 bits, it means that they don't satisfy $a \geq b$. The constraint pseudo code is
\begin{lstlisting}[language=Rust]
range_check(integer256_a, range256bit)
range_check(integer256_b, range256bit)
let diff = integer256_a - integer256_b
range_check(diff, range256bit)
\end{lstlisting}

We can combine \verb|GTE| and \verb|NEQ| with bitwise operations to perform other comparisons, listed in Table \ref{table:comparison-level}.
\begin{table}[!ht]
    \centering
    \begin{tabular}{|l|l|}
    \hline
        \emph{Application level} & \emph{Circuit level} \\ \hline
        $a \ge b$ & a \verb|GTE| b \\
        $a \le b$ & b \verb|GTE| a \\
        $a > b$ & (a \verb|GTE| b) \&\& (a \verb|NEQ| b) \\
        $a < b$ & (b \verb|GTE| a) \&\& (b \verb|NEQ| a) \\ \hline
    \end{tabular}
    \caption{Comparison level}
    \label{table:comparison-level}
\end{table}
