\subsection{Bitwise}

Lookup Arguments are also utilized to implement most of the bitwise operations. The basic lookup table is based on 8 bits, containing three columns, two inputs and one bitwise output, as shown in Table \ref{table:bitwise-lookup}.

\begin{table}[!ht]
    \centering
    \begin{tabular}{|l|l|l|l|}
    \hline
        & \emph{lhs} & \emph{rhs} & \emph{out} \\ \hline
        BitwiseAND table & lhs = 8-bit & rhs = 8-bit & lhs AND rhs \\ \hline
        BitwiseOR table & lhs = 8-bit & rhs = 8-bit & lhs OR rhs \\ \hline
        BitwiseXOR table & lhs = 8-bit & rhs = 8-bit & lhs XOR rhs \\ \hline
    \end{tabular}
    \caption{Bitwise lookup table}
    \label{table:bitwise-lookup}
\end{table}

Bitwise operations larger than 8 bits are broken down into multiple 8-bit operations with Plookup ran on each operation. Take the 32-bit AND of two numbers as an example,
\begin{align*}
a \land b &= (a_0 + a_1\cdot2^8 + a_2 \cdot (2^8)^2 + a_3\cdot(2^8)^3) \land (b_0 + b_1\cdot2^8 + b_2 \cdot (2^8)^2 + b_3\cdot(2^8)^3)\\ &= (a_0 \land b_0) + (a_1 \land b_1)\cdot2^8 + (a_2 \land b_2)\cdot(2^8)^2+(a_3 \land b_3)\cdot(2^8)^3.
\end{align*}

Therefore, for two $n$-bit numbers $a$ and $b$, let $m=n/8$, and $a_i,b_i$ are their limbs, then
\begin{align*}
a \land b &= \sum_{i=0}^{m-1} (2^{8i} \cdot (a_i \land b_i)), \\
a \lor b &= \sum_{i=0}^{m-1} (2^{8i} \cdot (a_i \lor b_i)), \\
a \oplus b &= \sum_{i=0}^{m-1} (2^{8i} \cdot (a_i \oplus b_i)).
\end{align*}

The implementation of NOT is slightly different. The NOT to constrain $a$ is
\[ \neg a + a = 2^{256} - 1. \]
