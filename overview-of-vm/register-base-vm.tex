\subsection{Register-based VM}

Among virtual machine architectures, stack-based and register-based are common implementations. Stack-based implementation is used by e.g, Java VM \cite{gunther2019scenery} and EVM \cite{website:evm}, and register-based virtual machine implementations by e.g, Lua VM \cite{ierusalimschy2005implementation} and Cairo VM \cite{cryptoeprint:2021/1063}. These two types of virtual machines adopt the mode of simulating physical CPU to execute instructions, but there are significant differences in execution efficiency and the speed of compiling and generating code in this specific implementation process. Take a simple addition instruction as an example to illustrate the difference between two virtual machine architectures in the process of executing instructions. The storage data structure of a stack-based virtual machine adopts the operation form of FIFO (First in, First out). First, the operand is popped out of the stack, and then the operation is performed, and the operation result will be pushed into the stack again. The instructions involved are
\begin{lstlisting}
POP (1)
POP (2)
ADD 1, 2, result
PUSH result
\end{lstlisting}
As can be seen in the addition operation, 4 instructions are required, which is relatively inefficient, however, the advantage is that you don't need to know the specific address of the operand, and you can get the operand by using instruction \verb|POP|.

Register-based VMs do not have such an operand stack to store operands. Each instruction will contain the register address of the operand. The above example involves the following instruction
\begin{lstlisting}
ADD R0, R1, R2
\end{lstlisting}

The obvious advantage of register-based virtual machines is that the addition can be completed with a single instruction. The disadvantage is that the address of the operand needs to be included in the instruction, which makes the instruction length longer.

Two virtual machine architectures each have their own advantages and disadvantages, further comparisons are shown in Table \ref{table:stacked-or-register}. Since our goal is to design a ZK-friendly virtual machine, we aim to achieve smaller execution traces of a given program with the trade-off increased instruction length. Furthermore, register-based virtual machines are easier to optimize during its execution of the instructions, hence, execution speed will see a significant improvement. Therefore we have opted for using a register-based virtual machine.
\begin{table}[!ht]
    \centering
    \begin{tabular}{|c|c|}
    \hline
        \emph{Stack-based VM vs. Register-based VM} & \emph{Comparison} \\ \hline
        Number of instructions & Register-based $<$ Stack-based \\
        Instruction length & Register-based $>$ Stack-based \\
        Difficulty of the code generation & Register-based $>$ Stack-based \\
        Instruction optimization & Register-based VM is easier to optimize code \\
        Execution speed & Register-based VMs have a faster execution pace \\ \hline
    \end{tabular}
    \caption{Stack-based VMs vs. Register-based VMs}
    \label{table:stacked-or-register}
\end{table}
