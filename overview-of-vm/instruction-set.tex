\subsection{Instruction Set} \label{subsec:instruction-set}

Whilst designing a ZKVM we also considered the complexity of verification, in addition to the execution efficiency of the VM itself, which differs from the traditional design principles of the instruction set. Purely pursuing efficient execution of instructions is not our goal, we have other fundamental design objectives, such as achieving short and concise verification complexity (occupying fewer trace cell units).

For example: when you have two instruction sets $A$ and $B$, there is an ``is\_zero" instruction in instruction set $A$, which is used to judge ``if x = 0, res = 1, otherwise, res = 0”, and except for this instruction, instruction set $B$ is identical to instruction set $A$. We denote the number of track units occupied by a CPU based on instruction set $A$ to execute an instruction by $a$ (for simplicity, assuming that all instructions consume the same number of units), and the number of track units required to execute one step based on instruction set $B$ by $b$ (where $a>b$ due to the added complexity of additional instructions). On the other hand, if a deterministic program is written based on instruction set $A$, it will execute $k_A$; and if it's written based on instruction set $B$, it will execute $k_B$ ($k_B>k_A$, it is possible to utilize more instructions to execute ``is\_zero”). If $a \cdot k_A < b \cdot k_B$, instruction set $A$ is more suitable for this program; conversely, instruction set $B$ is more suitable. When deciding whether to intervene in an instruction, we should consider the relationship between the additional cost of each step ($a/b$) and the additional steps added ($k_A/k_B$).

Based on the principles above, the instruction sets we designed are as shown in Table \ref{table:instruction-set}, \verb|flag| is a special register that stores the flag of \verb|overflow| or \verb|borrow|.
\begin{table}[!ht]
    \resizebox{\textwidth}{!}{
    \begin{tabular}{|c|c|c|c|c|c|}
    \hline
    \emph{Type} & \emph{Sub Type} & \emph{Instruction} & \emph{Operands} & \emph{Description} & \emph{flag} \\ \hline
     &  & ADD & r0 r1 r2 & Compute [r1] + [r2] and store the result in r0 & overflow \\ \cline{3-6} 
     &  & SUB & r0 r1 r2 & Compute [r1] - [r2] and store the result in r0 & borrow \\ \cline{3-6} 
     &  & MUL & r0 r1 r2 & Compute [r1] * [r2] and store the least significant bits of the result in r0 & overflow \\ \cline{3-6} 
     &  & DIV & r0 r1 r2 & Compute quotient of [r1]/[r2] and store result in r0 & [r2]= 0 \\ \cline{3-6} 
    \multirow{-5}{*}{Logic} & \multirow{-5}{*}{arithmetic} & MOD & r0 r1 r2 & Compute remainder of [r1] /[r2] and store result in r0 & [r2] = 0 \\ \hline
     &  & JMP & r0 & Set pc to [r0] &  \\ \cline{3-6} 
     & \multirow{-2}{*}{jump} & CJMP & r0 & If flag = 1, set pc to [r0], else increment pc as usual &  \\ \cline{2-6} 
    \multirow{-3}{*}{Flow} & end & END &  & END means the program is terminated and all registers are cleared &  \\ \hline
     & mov & MOV & r0 r1 & Store [r1] in r0 &  \\ \cline{2-6} 
    \multirow{-2}{*}{Move} & cmov & CMOV & r0 r1 & Store [r1] in r0, if flag = 1 &  \\ \hline
     & lt & LT & r0 r1 & \cellcolor[HTML]{F9FAFB}Less-than comparison & [r0] < r[1] \\ \cline{2-6} 
     & gt & GT & r0 r1 & Greater-than comparison & [r0] > r[1] \\ \cline{2-6} 
    \multirow{-3}{*}{Compare} & eq & EQ & r0 r1 & \cellcolor[HTML]{F9FAFB}Equality comparison & r[0] = r[1] \\ \hline
     & shl & SHL & r0 r1 r2 & shift [r1] by [r2] bits to the left and store result in r0 &  \\ \cline{2-6} 
    \multirow{-2}{*}{Shfit} & shr & SHR & r0 r1 r2 & shift [r1] by [r2] bits to the right and store result in r0 &  \\ \hline
     & not & NOT & r0 r1 & Compute bitwise NOT of [r1] and store result in r0 &  \\ \cline{2-6} 
     & xor & XOR & r0 r1 r2 & Compute bitwise XOR of [r1] and [r2] and store result in r0 &  \\ \cline{2-6} 
     & and & AND & r0 r1 r2 & Compute bitwise AND of [r1] and [r2] and store result in r0 &  \\ \cline{2-6} 
    \multirow{-4}{*}{Bit} & or & OR & r0 r1 r2 & Compute bitwise OR of [r1] and [r2] and store result in r0 &  \\ \hline
     &  & MLOAD & r0 r1 & Load the Word in memory begin with start index is [r1] and store the Word into r0 &  \\ \cline{3-6} 
     & \multirow{-2}{*}{memory} & MSTORE & r1 r0 & Store [r0] at the start of memory where the start index is [r1] &  \\ \cline{2-6} 
     &  & SLOAD & r0 r1 & \begin{tabular}[c]{@{}c@{}}Use [r1] as key, query the value of the key from storage\\ (the key will be hashed and prefixed with a splice) and store it in r0\end{tabular} &  \\ \cline{3-6} 
    \multirow{-4}{*}{RAM} & \multirow{-2}{*}{storage} & SSTORE & r1 r0 & \begin{tabular}[c]{@{}c@{}}Use [r1] as key, [r0] as value, store the key and the value into storage\\ (the key will be hashed and prefixed with a splice)\end{tabular} &  \\ \hline
    Input & input & READ & r0 & Read the external input from VM freeInput, and store one Word into r0 &  \\ \hline
    \end{tabular}
    }
    \caption{Instruction set}
    \label{table:instruction-set}
    \end{table}


\emph{Note:} We haven't designed comparison instructions because comparison operations can be implemented directly through multiple Range Checks based on the particular design and parameter selection of OlaVM. Specific principles of this will be introduced in later sections.
