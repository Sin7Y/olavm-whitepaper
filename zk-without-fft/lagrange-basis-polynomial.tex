\subsection{Lagrange Basis Polynomial}

According to Lagrange Interpolation Algorithm, given $n$ points $(x_0,y_0),\ldots,(x_{n-1},y_{n-1})$, there exists a unique polynomial with a degree less than $n$ passing through these $n$ interpolation points, which can be expressed as
\[ f(x) = \sum_{i=0}^{n-1} y_i \prod_{j \ne i} \frac{x-x_j}{x_i-x_j}. \]
Define Lagrange basis polynomial as
\[ L_i(x) = \prod_{j \ne i} \frac{x-x_j}{x_i-x_j}. \]
Then the above formula can be written as $f(x)=\sum_{i=0}^{n-1}y_iL_i(x)$. Let $v(x)=\prod_{i=0}^{n-1}(x-x_i)$, using the formal derivative of $v(x)$
\begin{align*}
    v'(x) &= \sum_{i=0}^{n-1}\prod_{j \ne i}(x_i-x_j), \\
    v'(x_i) &= \prod_{j \ne i}(x_i-x_j).
\end{align*}
we can simplify the Lagrange basis to
\[ L_i(x) = \prod_{j \ne i}\frac{x-x_j}{x_i-x_j} = \frac{v(x)}{v'(x)(x-x_i)}. \]
All polynomials with a degree less than $n$, form an $n$-dimensional vector space under polynomial addition. For a polynomial $f(x)$ of a degree less than $n$, \[ f(x) = \sum_{i=0}^{n-1} a_ix^i, \]
$1,x,\ldots,x^{n-1}$ is a basis of the vector space, and the coordinate $a_i$ is called the coefficient form of $f(x)$. On the other hand, \[ f(x) = \sum_{i=0}^{n-1}f(x_i)L_i(x), \]
Lagrange basis $L_i(x)$ is also a basis of the vector space, and the coordinate $f_i=f(x_i)$ is called the evaluation form of $f(x)$.
The FFT/IFFT algorithm enables the transformation of the coordinates between the two bases.

Next, we will study how to calculate {\Update the} exact division when the divisor is a polynomial of degree 1, i.e.\ given $f(a)=0$, calculate the value of $q(x)=f(x)/(x-a)$ at the interpolation point $x_i$.
When $x_m \ne a$, $q(x_m)=f_m/(x_m-a)$, when $x_m=a$, since the degree of $q(x)=f(x)/(x-x_m)$ is less than $n-1$, we only need $n-1$ values of interpolation points $x_0,\ldots,x_{m-1} ,x_{m+1},\ldots,x_{n-1}$ to uniquely determine $q(x)$.
\begin{align*}
    q(x) &= \frac{f(x)}{x-x_m} \\
    &= \sum_{i \ne m} \frac{f_i}{x_i-x_m} \prod_{{\Update \substack{j \ne i \\ j \ne m}}} \frac{x-x_j}{x_i-x_j} \\
    &= \sum_{i \ne m} \frac{f_i}{x-x_m} \prod_{j \ne i} \frac{x-x_j}{x_i-x_j} \\
    &= \sum_{i \ne m} \frac{f_i}{x-x_m} L_i(x) \\
    &= \sum_{i \ne m} \frac{f_i}{x-x_m} \frac{v(x)}{v'(x)(x-x_i)} \\
    &= \sum_{i \ne m} \frac{f_i}{v'(x_i)(x-x_i)} \cdot \prod_{i \ne m}(x-x_i).
\end{align*}
Then
\begin{align*}
    q_m = q(x_m) &= \sum_{i \ne m} \frac{f_i}{v'(x_i)(x_m-x_i)} \cdot \prod_{i \ne m}(x_m-x_i) \\
    &= \sum_{i \ne m} \frac{f_i}{x_m-x_i} \cdot \frac{v'(x_m)}{v'(x_i)}.
\end{align*}

For non-interpolation points $z \notin \{x_0, \ldots, x_{n-1}\}$, we can calculate its value {\Update easily}
\begin{align*}
    f(z) = \sum_{i=0}^{n-1}f_iL_i(z)
    &= \sum_{i=0}^{n-1}\frac{v(z)f_i}{v'(z)(z-x_i)} \\
    &= v(z) \sum_{i=0}^{n-1}\frac{f_i}{v'(x_i)(z-x_i)}.
\end{align*}
If the interpolation point takes the $n$-th root of unity: $x_i=\omega^i$, then we have
\[ v(x)=x^n-1, \qquad v'(x)=nx^{n-1}, \qquad v'(x_i)=n\omega^{-i}, \]
\[ q_m = \sum_{i \ne m} \frac{f_i}{\omega^m-\omega^i} \cdot \frac{n\omega^{-m}}{n\omega^{-i}}
= \sum_{i \ne m} \frac{\omega^{i-m} f_i}{\omega^m-\omega^i}, \]
\[ f(z) = (z^n-1) \sum_{i=0}^{n-1}\frac{f_i}{n\omega^{-i}(z-\omega^i)}
= \frac{z^n-1}{n}\sum_{i=0}^{n-1}\frac{f_i\omega^i}{z-\omega^i}. \]
