\section{Structure of Execution Trace Table} \label{sec:structure-of-execution-trace-table}

As an initial step, by dividing the instructions, data and other contents involved in the program execution trace into one main trace and multiple sub traces we are able to prove that the program is being executed correctly. There is a hierarchical relationship between these sub traces and the main trace. When there are instructions that cannot be processed in the main trace, these instructions will be executed in a sub trace. These sub traces are associated with the main trace through Lookup Argument. The complexity of polynomial representation of the main trace table has been reduced and the speed of proof generation is improved through this division of main and sub traces.

This section primarily introduces the main trace table and the RAM trace table, and others like arithmetic sub trace table, binary sub trace table will be introduced in other sections.

In this section you'll find a description of the main trace table and RAM trace table, before that we've listed a few variables that will be referenced to. Some variables that have been described in Section \ref{subsec:instruction-structure} will not be repeated below.
\begin{itemize}
    \item \verb|clk|: The global clock of the program execution tracking. Every time an instruction is executed, \verb|clk| automatically increases by 1;
    \item \verb|instruction|: The encoded representation of an instruction;
    \item \verb|pc|: A program counter. Under normal circumstances, the \verb|pc| update logic is: \verb|pc = pc + instruction_size|, in other cases, such as \verb|flow_op|, the \verb|pc| will be changed according to the corresponding logic;
    \item \verb|fp|: Special register used in \verb|CALL| and \verb|RET| instructions, stores a pointer to the RPR memory location;
    \item \verb|const|: Used to store the immediate value of the instruction;
    \item \verb|r0|, \verb|r1|, \verb|r2|: Three general-purpose registers used for the storage of instruction input and output data;
    \item \verb|carry|: Indicates whether there is a carry;
    \item \verb|address|: The address operated by RAM instructions;
    \item \verb|value|: The value operated by RAM instructions;
    \item \verb|oldRoot|: World state representation of Verkle Tree before update;
    \item \verb|newRoot|: World state representation of Verkle Tree after update.
\end{itemize}

\subsection{Main Trace Table}

In this section we are going to explain how the main trace table is generated, below you'll find a simplified execution program that we will be using as an example in order to walk you through the process.

\begin{lstlisting}
READ R2(7)
MOV R1 3
ADD R0, R1, R2
END
\end{lstlisting}

This program would generate the main trace table as shown in Table \ref{table:main-trace-table}.
\begin{table}[!ht]
    \resizebox{\textwidth}{!}{
    \begin{tabular}{|c|c|c|c|c|c|c|c|c|c|c|c|c|}
    \hline
    \multirow{2}{*}{clk} & instruction & arith\_op & flow\_op & move\_op & ram\_op & input\_op & pc\_update & fp\_update & ext\_input & const\_input & r\_input & r\_output \\ \cline{2-13}
    & tape & pc & fp & const & r0 & r1 & r2 & carry & address & value & oldRoot & newRoot \\ \hline
    \multirow{2}{*}{0} & 00000000000000010000000010000100 & 0000 & 00000 & 00 & 0000 & 1 & 00000 & 000 & 1 & 0 & 000 & 100 \\ \cline{2-13}
    & 7 & 0 & 0x00 & 0 & 0 & 0 & 0 & 0 & 0x00 & 0 & 0x00 & 0x00 \\ \hline
    \multirow{2}{*}{1} & 00000000010000000000000001000010 & 0000 & 00000 & 10 & 0000 & 0 & 00000 & 000 & 0 & 1 & 000 & 010 \\ \cline{2-13}
    & 0 & 1 & 0x00 & 3 & 0 & 0 & 0 & 0 & 0x00 & 0 & 0x00 & 0x00 \\ \hline
    \multirow{2}{*}{2} & 00010000000000000000000000110001 & 0001 & 00000 & 00 & 0000 & 0 & 00000 & 000 & 0 & 0 & 110 & 001 \\ \cline{2-13}
    & 0 & 2 & 0x00 & 0 & 0 & 3 & 7 & 0 & 0x00 & 0 & 0x00 & 0x00 \\ \hline
    \multirow{2}{*}{3} & 00001000000000001000000000000111 & 0000 & 10000 & 00 & 0000 & 0 & 10000 & 000 & 0 & 0 & 000 & 111 \\ \cline{2-13}
    & 0 & 3 & 0x00 & 0 & 10 & 3 & 7 & 0 & 0x00 & 0 & 0x00 & 0x00 \\ \hline
    \multirow{2}{*}{4} & $\cdots$ & $\cdots$ & $\cdots$ & $\cdots$ & $\cdots$ & $\cdots$ & $\cdots$ & $\cdots$ & $\cdots$ & $\cdots$ & $\cdots$ & $\cdots$ \\ \cline{2-13}
    & $\cdots$ & $\cdots$ & $\cdots$ & $\cdots$ & $\cdots$ & $\cdots$ & $\cdots$ & $\cdots$ & $\cdots$ & $\cdots$ & $\cdots$ & $\cdots$ \\ \hline
    \end{tabular}
    }
    \caption{Main trace table}
    \label{table:main-trace-table}
\end{table}

The main trace table demonstrates the changes made in the main state machine. Starting from its initial state, a new state is generated after each instruction is executed. Whenever \verb|clk| changes, the state of the last executed instruction will have some connections to the currently executed instruction. Taking \verb|pc| column as an example, we regard each value appearing in \verb|pc| column as the evaluation of a polynomial, e.g.\ $\omega^0$ is 0, $\omega^1$ is 1, $\omega^2$ is 2 and so on. Similarly, each column can be regarded as a polynomial. Finally, we only need to combine these polynomials together to prove the correctness of the main trace table, thus proving that the program is executed correctly.

\subsection{RAM Trace Table}

In order to facilitate the correctness constraints for memory and storage operations, we utilize sub traces to represent all read and write operations for memory and storage. Since the state machines of memory and storage are very similar, we can combine them together, called RAM trace collectively, and distinguish between memory/storage operations through different \verb|ram_ops|. Table \ref{table:ram-trace-table} demonstrates the RAM trace table of a simple example program.
\begin{table}[!ht]
    \centering
    \begin{tabular}{|c|c|c|c|c|c|}
    \hline
    \emph{clk} & \emph{ram\_op} & \emph{Read} & \emph{Write} & \emph{Address} & \emph{Value} \\ \hline
    1 & MSTORE & 0 & 1 & \texttt{0x80} & value1 \\
    2 & SSTORE & 0 & 1 & \texttt{0x7f22} & value2 \\
    3 & SLOAD & 1 & 0 & \texttt{0x7f22} & value2 \\
    4 & MLOAD & 1 & 0 & \texttt{0x80} & value1 \\
    $\cdots$ & $\cdots$ & $\cdots$ & $\cdots$ & $\cdots$ & $\cdots$ \\ \hline
    \end{tabular}
    \caption{RAM trace table}
    \label{table:ram-trace-table}
\end{table}

The above RAM trace table shows the memory read and write operations of one address and the storage read and write operations of another address. All instructions are listed in the order of \verb|clk|, read and write operations are represented by separate columns. Considering that the VM will randomly read and write memory and storage during the execution process of reading and writing, it's difficult to make a unified circuit constraint in this case (it would require backtracking and recording the operation process continuosly for each tag, finally receiving all operations for a specific tag). This leads us to sort the RAM trace table in a specific order, which is: \verb|ram_op -> address -> clk|. The sorted RAM trace table is shown in Table \ref{table:sorted-ram-trace-table}.
\begin{table}[!ht]
    \centering
    \begin{tabular}{|c|c|c|c|c|c|}
    \hline
    \emph{clk} & \emph{ram\_op} & \emph{Read} & \emph{Write} & \emph{Address} & \emph{Value} \\ \hline
    1 & MSTORE & 0 & 1 & \texttt{0x80} & value1 \\
    4 & MLOAD & 1 & 0 & \texttt{0x80} & value1 \\
    2 & SSTORE & 0 & 1 & \texttt{0x7f22} & value2 \\
    3 & SLOAD & 1 & 0 & \texttt{0x7f22} & value2 \\
    $\cdots$ & $\cdots$ & $\cdots$ & $\cdots$ & $\cdots$ & $\cdots$ \\ \hline
    \end{tabular}
    \caption{Sorted RAM trace table}
    \label{table:sorted-ram-trace-table}
\end{table}

Some additional circuit constraints required for the RAM trace table are shown below:
\begin{itemize}
    \item The value of read and write columns must be of the Boolean type, read as 0, write as 1.
    \item \verb|ram_op| operation must have only 4 types, namely MSTORE, MLOAD, SSTORE, SLOAD, among which MSTORE, MLOAD are classified as the same type of tags and tags of the same type are not sorted.
    \item RAM trace table must be sorted according to the rules \[ \text{ram\_op}(\text{MSTORE/MLOAD})_{i} \le \text{ram\_op}(\text{SSTORE/SLOAD})_{i+1}\] When tags are the same, the sorting rules are: $\text{address}_{i} \le \text{address}_{i+1}$; when addresses are the same, the sorting rules are: $\text{clk}_{i} \le \text{clk}_{i+1}$.
    \item For an operation of the same address, if it is a read operation, the value read must be the last value written to the address.
\end{itemize}

