\subsection{Main Trace Table}

In this section we are going to explain how the main trace table is generated, below you'll find a simplified execution program that we will be using as an example in order to walk you through the process.

\begin{lstlisting}
READ R2(7)
MOV R1 3
ADD R0, R1, R2
END
\end{lstlisting}

This program would generate the main trace table as shown in Table \ref{table:main-trace-table}.
\begin{table}[!ht]
    \resizebox{\textwidth}{!}{
    \begin{tabular}{|c|c|c|c|c|c|c|c|c|c|c|c|c|}
    \hline
    \multirow{2}{*}{clk} & instruction & arith\_op & flow\_op & move\_op & ram\_op & input\_op & pc\_update & fp\_update & ext\_input & const\_input & r\_input & r\_output \\ \cline{2-13}
    & tape & pc & fp & const & r0 & r1 & r2 & carry & address & value & oldRoot & newRoot \\ \hline
    \multirow{2}{*}{0} & 00000000000000010000000010000100 & 0000 & 00000 & 00 & 0000 & 1 & 00000 & 000 & 1 & 0 & 000 & 100 \\ \cline{2-13}
    & 7 & 0 & 0x00 & 0 & 0 & 0 & 0 & 0 & 0x00 & 0 & 0x00 & 0x00 \\ \hline
    \multirow{2}{*}{1} & 00000000010000000000000001000010 & 0000 & 00000 & 10 & 0000 & 0 & 00000 & 000 & 0 & 1 & 000 & 010 \\ \cline{2-13}
    & 0 & 1 & 0x00 & 3 & 0 & 0 & 0 & 0 & 0x00 & 0 & 0x00 & 0x00 \\ \hline
    \multirow{2}{*}{2} & 00010000000000000000000000110001 & 0001 & 00000 & 00 & 0000 & 0 & 00000 & 000 & 0 & 0 & 110 & 001 \\ \cline{2-13}
    & 0 & 2 & 0x00 & 0 & 0 & 3 & 7 & 0 & 0x00 & 0 & 0x00 & 0x00 \\ \hline
    \multirow{2}{*}{3} & 00001000000000001000000000000111 & 0000 & 10000 & 00 & 0000 & 0 & 10000 & 000 & 0 & 0 & 000 & 111 \\ \cline{2-13}
    & 0 & 3 & 0x00 & 0 & 10 & 3 & 7 & 0 & 0x00 & 0 & 0x00 & 0x00 \\ \hline
    \multirow{2}{*}{4} & $\cdots$ & $\cdots$ & $\cdots$ & $\cdots$ & $\cdots$ & $\cdots$ & $\cdots$ & $\cdots$ & $\cdots$ & $\cdots$ & $\cdots$ & $\cdots$ \\ \cline{2-13}
    & $\cdots$ & $\cdots$ & $\cdots$ & $\cdots$ & $\cdots$ & $\cdots$ & $\cdots$ & $\cdots$ & $\cdots$ & $\cdots$ & $\cdots$ & $\cdots$ \\ \hline
    \end{tabular}
    }
    \caption{Main trace table}
    \label{table:main-trace-table}
\end{table}

The main trace table demonstrates the changes made in the main state machine. Starting from its initial state, a new state is generated after each instruction is executed. Whenever \verb|clk| changes, the state of the last executed instruction will have some connections to the currently executed instruction. Taking \verb|pc| column as an example, we regard each value appearing in \verb|pc| column as the evaluation of a polynomial, e.g.\ $\omega^0$ is 0, $\omega^1$ is 1, $\omega^2$ is 2 and so on. Similarly, each column can be regarded as a polynomial. Finally, we only need to combine these polynomials together to prove the correctness of the main trace table, thus proving that the program is executed correctly.
