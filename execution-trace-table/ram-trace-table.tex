\subsection{RAM Trace Table}

In order to facilitate the correctness constraints for memory and storage operations, we utilize sub traces to represent all read and write operations for memory and storage. Since the state machines of memory and storage are very similar, we can combine them together, called RAM trace collectively, and distinguish between memory/storage operations through different \verb|ram_ops|. Table \ref{table:ram-trace-table} demonstrates the RAM trace table of a simple example program.
\begin{table}[!ht]
    \centering
    \begin{tabular}{|c|c|c|c|c|c|}
    \hline
    \emph{clk} & \emph{ram\_op} & \emph{Read} & \emph{Write} & \emph{Address} & \emph{Value} \\ \hline
    1 & MSTORE & 0 & 1 & \texttt{0x80} & value1 \\
    2 & SSTORE & 0 & 1 & \texttt{0x7f22} & value2 \\
    3 & SLOAD & 1 & 0 & \texttt{0x7f22} & value2 \\
    4 & MLOAD & 1 & 0 & \texttt{0x80} & value1 \\
    $\cdots$ & $\cdots$ & $\cdots$ & $\cdots$ & $\cdots$ & $\cdots$ \\ \hline
    \end{tabular}
    \caption{RAM trace table}
    \label{table:ram-trace-table}
\end{table}

The above RAM trace table shows the memory read and write operations of one address and the storage read and write operations of another address. All instructions are listed in the order of \verb|clk|, read and write operations are represented by separate columns. Considering that the VM will randomly read and write memory and storage during the execution process of reading and writing, it's difficult to make a unified circuit constraint in this case (it would require backtracking and recording the operation process continuosly for each tag, finally receiving all operations for a specific tag). This leads us to sort the RAM trace table in a specific order, which is: \verb|ram_op -> address -> clk|. The sorted RAM trace table is shown in Table \ref{table:sorted-ram-trace-table}.
\begin{table}[!ht]
    \centering
    \begin{tabular}{|c|c|c|c|c|c|}
    \hline
    \emph{clk} & \emph{ram\_op} & \emph{Read} & \emph{Write} & \emph{Address} & \emph{Value} \\ \hline
    1 & MSTORE & 0 & 1 & \texttt{0x80} & value1 \\
    4 & MLOAD & 1 & 0 & \texttt{0x80} & value1 \\
    2 & SSTORE & 0 & 1 & \texttt{0x7f22} & value2 \\
    3 & SLOAD & 1 & 0 & \texttt{0x7f22} & value2 \\
    $\cdots$ & $\cdots$ & $\cdots$ & $\cdots$ & $\cdots$ & $\cdots$ \\ \hline
    \end{tabular}
    \caption{Sorted RAM trace table}
    \label{table:sorted-ram-trace-table}
\end{table}

Some additional circuit constraints required for the RAM trace table are shown below:
\begin{itemize}
    \item The value of read and write columns must be of the Boolean type, read as 0, write as 1.
    \item \verb|ram_op| operation must have only 4 types, namely MSTORE, MLOAD, SSTORE, SLOAD, among which MSTORE, MLOAD are classified as the same type of tags and tags of the same type are not sorted.
    \item RAM trace table must be sorted according to the rules \[ \text{ram\_op}(\text{MSTORE/MLOAD})_{i} \le \text{ram\_op}(\text{SSTORE/SLOAD})_{i+1}\] When tags are the same, the sorting rules are: $\text{address}_{i} \le \text{address}_{i+1}$; when addresses are the same, the sorting rules are: $\text{clk}_{i} \le \text{clk}_{i+1}$.
    \item For an operation of the same address, if it is a read operation, the value read must be the last value written to the address.
\end{itemize}
