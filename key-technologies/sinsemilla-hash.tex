\subsection{Sinsemilla Hash}

Sinsemilla Hash is a Lookup Argument-friendly hash algorithm based on the complexity of discrete logarithm problem and collision resistance (the length of input is fixed). Compared with other algebraic hash algorithms, such as Rescue and Poseiden, Sinsemilla Hash has certain advantages and disadvantages:
\begin{itemize}
    \item Advantage: Sinsemilla Hash executes 19 times faster than Rescue and Poseiden Hash when executed out of circuit;
    \item Disadvantage: When executing with circuits, Sinsemilla Hash is 4 times slower than Rescue and Poseiden Hash.
\end{itemize}
Considering the local execution of hash calculation when generating the execution trace, Sinsemilla Hash is a better option from the perspective of the overall performance (ranging from data preparation to proof generation). Next let's take a look at the algorithm details of Sinsemilla Hash.

\subsubsection{Setup}

\begin{enumerate}
    \item Select the split parameter $k$;
    \item Generate $2^k+1$ independent points $\{Q, P_0, P_1, \ldots, P_{2^k-1}\}$.
\end{enumerate}

\subsubsection{Hash(M)}

\begin{enumerate}
    \item Split $M$ into $n$ groups of $k$ bits, the $i$-th group is represented as $m_i$ in little endian;
    \item Set the initial value $\mathrm{Acc}_0 = Q$.
    \item For $i$ from 0 to $n-1$, execute loop calculation $\mathrm{Acc}_{i+1} = (\mathrm{Acc}_i + P_{m_{i+1}}) + \mathrm{Acc}_i$.
    \item Finally we get $\mathrm{Acc}_n$.
    \item $\mathrm{ShortHash}(M)$ returns to the $x$-coordinate of $\mathrm{Hash}(M)$.
\end{enumerate}

\subsubsection{Incomplete Addition}

In step 3 of $\mathrm{Hash}(M)$, we have performed operation $\mathrm{Acc}_{i+1} = (\mathrm{Acc}_i + P_{m_{i+1}}) + \mathrm{Acc}_i$, where the elliptic curve addition algorithm used here is Incomplete addition, i.e., it can only process elliptic curve addition in valid input scenarios (the inputs are different points on the elliptic curve). The other one is a complete addition. The specific differences of the two calculation forms are shown in Incomplete and complete addition -- The halo2 Book \cite{website:halo2}.
\begin{itemize}
    \item Input: $P=(x_P,y_P)$, $Q=(x_Q,y_Q)$,
    \item Output: $R=P+Q=(x_R,y_R)$.
\end{itemize}

According to the introduction in Section 4.1 of \cite{quteprints33233}, the Incomplete addition formula can be expressed as
\begin{align*}
    x_R &= \left(\frac{y_Q-y_P}{x_Q-x_P}\right)^2 - x_Q - x_P, \\
    y_R &= \frac{y_Q-y_P}{x_Q-x_P} \cdot (x_Q - x_R) - y_Q.
\end{align*}

According to the definition of Incomplete addition, we have $x_Q \ne x_P$, so the previous formula can be transformed into
\begin{align*}
    & x_R = \left(\frac{y_Q-y_P}{x_Q-x_P}\right)^2 - x_Q-x_P \\
    \implies & x_R + x_Q + x_P = \left(\frac{y_Q-y_P}{x_Q-x_P}\right)^2 \\
    \implies & (x_R + x_Q + x_P)(x_Q - x_P)^2 - (y_Q - y_P)^2 = 0, \\
    & y_R = \frac{y_Q-y_P}{x_Q-x_P} \cdot (x_Q - x_R) - y_Q \\
    \implies & (y_R + y_Q)(x_Q - x_P) - (y_Q - y_P)(x_Q - x_R) = 0.
\end{align*}

So, for Incomplete addition of elliptic curve points, we have the following constraints:
\begin{table}[!ht]
    \centering
    \begin{tabular}{|c|c|}
        \hline
        \emph{Degree} & \emph{Constraint} \\
        \hline
        4 & $q_\text{incomplete-add} \cdot ((x_R + x_Q + x_P)(x_Q - x_P)^2 - (y_Q - y_P)^2) = 0$ \\
        3 & $q_\text{incomplete-add} \cdot ((y_R + y_Q)(x_Q - x_P) - (y_Q - y_P)(x_Q - x_R)) = 0$ \\
        \hline
    \end{tabular}
    \caption{Constraints for incomplete addition}
\end{table}

Let $\lambda=(y_Q-y_P)/(x_Q-x_P)$, the above constraints can be changed to:
\begin{table}[!ht]
    \centering
    \begin{tabular}{|c|c|}
        \hline
        \emph{Degree} & \emph{Constraint} \\
        \hline
        3 & $q_\text{incomplete-add} \cdot(x_R + x_Q + x_P - \lambda^2) = 0$ \\
        3 & $q_\text{incomplete-add} \cdot(\lambda(x_Q - x_R) - (y_Q + y_R)) = 0$ \\
        3 & $q_\text{incomplete-add} \cdot(\lambda(x_Q - x_P) - (y_Q - y_P)) = 0$ \\
        \hline
    \end{tabular}
    \caption{Improved constraints for incomplete addition}
\end{table}

\subsubsection{Lookup for Sinsemilla}

According to the calculation process of Sinsemilla, we need to split the original information $M$ into $n$ sub-information $m_i$ with a bit width of $k$ (since Sinsemilla has fixed-length input requirements, it's necessary to pad $M$ to a fixed length). Let $P_{m_i} = (x_{P,i}, y_{P,i})$, since $m_i \in [0,2^k)$, $k$ will not be large in general, then we can precompute a table for query as Table \ref{table:lookup-table-for-sinsemilla-hash}.

\begin{table}[!ht]
    \centering
    \begin{tabular}{|c|c|c|}
        \hline
        $i$ & $x$-coordinate & $y$-coordinate \\
        \hline
        $0$ & $x_{P,0}$ & $y_{P,0}$ \\
        $1$ & $x_{P,1}$ & $y_{P,1}$ \\
        $\cdots$ & $\cdots$ & $\cdots$ \\
        $2^k-1$ & $x_{P,2^k-1}$ & $y_{P,2^k-1}$ \\
        \hline
    \end{tabular}
    \caption{Lookup table for Sinsemilla Hash}
    \label{table:lookup-table-for-sinsemilla-hash}
\end{table}
