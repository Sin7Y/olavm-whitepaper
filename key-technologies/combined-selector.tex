\subsection{Combined Selector}

This section mainly introduces how to improve the efficiency of ZK by combining selectors. First up, it introduces why selectors appear, and then how to combining multiple selectors.

\subsubsection{Custom Gate}

When designing ZKVM circuit, many binary selectors are introduced due to the vast amount of custom gates. Looking at the field division gate as an example, we plan to design a gate to verify the equation $q = x/y$ between three field elements $q,x,y$. For convenience, we will not implement field division operation at circuit level, but by checking the equations
\begin{align*}
& x \cdot y^{-1} = q, \\
& y^{-1} \cdot y = 1, \quad (\text{ensure $y \ne 0$}).
\end{align*}
They are equivalent, so we have the following trace table:
\begin{table}[!ht]
\centering
\begin{tabular}{|c|c|c|c|c|c|c|c|c|c|}
\hline
clk & $s_*$ & $\cdots$ & $s_\text{div}$ & $\cdots$ & $w_0$ & $w_1$ & $w_2$ & $w_3$ & Notes \\
\hline
$\cdots$ & $\cdots$ & $\cdots$ & $\cdots$ & $\cdots$ & $\cdots$ & $\cdots$ & $\cdots$ & $\cdots$ & $\cdots$ \\
\hline
$k$ & 0 & 0 & 1 & 0 & $x$ & $y$ & $q$ & $y^{-1}$ & division \\
\hline
$\cdots$ & $\cdots$ & $\cdots$ & $\cdots$ & $\cdots$ & $\cdots$ & $\cdots$ & $\cdots$ & $\cdots$ & $\cdots$ \\
\hline
\end{tabular}
\caption{A simple example of execution trace}
\end{table}

For columns $w_0,w_1,w_2,w_3$, define polynomials $w_0(x),w_1(x),w_2(x),w_3(x)$, then on the row corresponding to the division operation, they should satisfy
\begin{align*}
w_0(x) \cdot w_3(x) - w_2(x) &= 0, \\
w_1(x) \cdot w_3(x) - 1 &= 0.
\end{align*}

In order to ensure the satisfaction of the above relationships on the corresponding row, it is necessary to enable the corresponding division to constrain the verification through a selector, for which we introduce a new column, like $s_\text{div}= \{0,0,\ldots,1,\ldots,0\}$. After converting to polynomial $s_\text{div}(x)$, the above equations become
\begin{align*}
s_\text{div}(x)\cdot(w_0(x) \cdot w_3(x) - w_2(x)) &= 0, \\
s_\text{div}(x)\cdot(w_1(x) \cdot w_3(x) - 1) &= 0.
\end{align*}

\subsubsection{Combined Selector}

According to the above example, we know that whenever we define a new custom gate, we need to introduce a selector column $s_*$ related to the gate, called $t_*(x)$, then we have the following constraints
\begin{align*}
s_\text{add}(x) \cdot t_\text{add}(x) &= 0, \\
s_\text{div}(x) \cdot t_\text{div}(x) &= 0, \\
s_\text{cube}(x) \cdot t_\text{cube}(x) &= 0, \\
s_\text{sqrt}(x) \cdot t_\text{sqrt}(x) &= 0.
\end{align*}

Since the prover needs to make commitments to all polynomials when generating proofs, the introduction of too many selector polynomials will increase the workload of both the prover and verifier. Therefore, we need to optimize the number of selectors, which requires to meet two conditions:
\begin{enumerate}
    \item Without losing the meaning of selector polynomials, i.e.\ only specific gates can be allowed;
    \item Fewer selector polynomials.
\end{enumerate}

Plonky2 \cite{website:plonky2} shows a Binary-Tree based Selector optimization method, which reduces the number of selector polynomials to $\log(k)$, where $k$ is the number of custom gates; in Halo2 \cite{website:halo2}, ZCash team shares a new optimization method that may achieve a smaller number of polynomials (which is related to the constraint polynomial $t_*(x)$ and the parameter \verb|max_degree| set to the constraint polynomial).

We've taken forth a simple scenario to ease the understanding (for detailed algorithm, please refer to Selector Combining \cite{website:selector-combining}).

\begin{table}[!ht]
\centering
\begin{tabular}{|c|c|c|c|c|c|c|c|c|c|}
  \hline
  clk & $s_\text{add}$ & $s_\text{div}$ & $s_\text{cube}$ & $s_\text{sqrt}$ & $w_0$ & $w_1$ & $w_2$ & $w_3$ & Notes \\
  \hline
  0 & 1 & 0 & 0 & 0 & $a_0$ & $b_0$ & $c_0$ & $d_0$ & addition \\
  1 & 0 & 1 & 0 & 0 & $a_1$ & $b_1$ & $c_1$ & $d_1$ & division \\
  2 & 0 & 0 & 1 & 0 & $a_2$ & $b_2$ & $c_2$ & $d_2$ & cube \\
  3 & 0 & 0 & 0 & 1 & $a_3$ & $b_3$ & $c_3$ & $d_3$ & sqrt \\
  \hline
\end{tabular}
\caption{Multi-operation example of execution trace}
\end{table}

As can be seen, we have set 4 selector columns for the four custom gates, which are not what we want, thus increasing the workload of the prover and verifier. Then we define a new column $q$, satisfying
\begin{align*}
q = \begin{cases}
   k & \text{if the selector labelled } k \text{ is } 1 \\
   0 & \text{if all the selectors are } 0
\end{cases}.
\end{align*}

If we define a set $\{s_\text{add},s_\text{div},s_\text{cube},s_\text{sqrt}\}$ for selectors $s_\text{add},s_\text{div},s_\text{cube},s_\text{sqrt}$ (we call this set disjoint, because the rows enabled are not intersecting), combining the definition of column $q$, we have
\begin{table}[!ht]
\centering
\begin{tabular}{|c|c|c|c|c|c|c|c|c|c|c|}
  \hline
  clk & $s_\text{add}$ & $s_\text{div}$ & $s_\text{cube}$ & $s_\text{sqrt}$ & $q$ & $w_0$ & $w_1$ & $w_2$ & $w_3$ & Notes \\
  \hline
  0 & 1 & 0 & 0 & 0 & 1 & $a_0$ & $b_0$ & $c_0$ & $d_0$ & addition \\
  1 & 0 & 1 & 0 & 0 & 2 & $a_1$ & $b_1$ & $c_1$ & $d_1$ & division \\
  2 & 0 & 0 & 1 & 0 & 3 & $a_2$ & $b_2$ & $c_2$ & $d_2$ & cube \\
  3 & 0 & 0 & 0 & 1 & 4 & $a_3$ & $b_3$ & $c_3$ & $d_3$ & sqrt \\
  \hline
\end{tabular}
\caption{Add combined column $q$}
\end{table}

Then we define a new selector polynomial form based on column $q$. The $k$-th selector polynomial is
\[ q(x)\prod_{\substack{h=1 \\ h \ne k}}^{\mathrm{len}(\text{selectors})}(h - q(x)). \]
For example, for constraint $s_\text{add} \cdot t_\text{add}(x) = 0$, we can rewrite it as
\[ q(x)\prod_{\substack{h=1 \\ h \ne 1}}^{\mathrm{len}(\text{selectors})}(h - q(x)) \cdot t_\text{add}(x) = 0. \]
The above equation can be expanded into
\[ q(x)\cdot (2- q(x))\cdot(3-q(x))\cdot (4-q(x))\cdot t_\text{add}(x) = 0. \]
Define
\[ \mathrm{combined}_q(x) = q(x)\cdot (2- q(x))\cdot(3-q(x))\cdot (4-q(x)), \]
then we can derive its truth table:
\begin{table}[!ht]
    \centering
    \begin{tabular}{|c|c|c|}
      \hline
      clk & $\mathrm{combined}_q(x)$ & $s_\text{add}(x)$ \\
      \hline
      0 & 1 & 1 \\
      1 & 0 & 0 \\
      2 & 0 & 0 \\
      3 & 0 & 0 \\
      \hline
    \end{tabular}
    \caption{Consistency between $q$ and $s_\text{add}$}
\end{table}

This achieves the same function as the original selector, therefore, for constraints
\begin{align*}
s_\text{add}(x) \cdot t_\text{add}(x) &= 0, \\
s_\text{div}(x) \cdot t_\text{add}(x) &= 0, \\
s_\text{cube}(x) \cdot t_\text{cube}(x) &= 0, \\
s_\text{sqrt}(x) \cdot t_\text{sqrt}(x) &= 0.
\end{align*}
we can rewrite them as
\begin{align*}
q(x)\prod_{\substack{h=1 \\ h \ne 1}}^{\mathrm{len}(\text{selectors})}(h - q(x)) \cdot t_\text{add}(x) &= 0, \\
q(x)\prod_{\substack{h=1 \\ h \ne 2}}^{\mathrm{len}(\text{selectors})}(h - q(x)) \cdot t_\text{add}(x) &= 0, \\
q(x)\prod_{\substack{h=1 \\ h \ne 3}}^{\mathrm{len}(\text{selectors})}(h - q(x)) \cdot t_\text{cube}(x) &= 0, \\
q(x)\prod_{\substack{h=1 \\ h \ne 4}}^{\mathrm{len}(\text{selectors})}(h - q(x)) \cdot t_\text{sqrt}(x) &= 0.
\end{align*}

For constraints of the types above, we only need to make commitments to the polynomial $q(x)$. But we should note that this method also increases the degree of constraints.

So far, we have achieved the two points mentioned above:
\begin{enumerate}
    \item Without losing the meanings of selector polynomials, i.e.\ only specific gates can be allowed;
    \item Fewer selector polynomials.
\end{enumerate}

Of course, due to the limitation of degree of constraints in the protocol, the selectors that can be combined are limited. The number of selectors combined at a time depends on the degree of the constraint polynomial $t_*(x)$ and the boundary value \verb|max_degree|. Therefore, we may need multiple combined columns, even so the number of which is much smaller than the original number.
