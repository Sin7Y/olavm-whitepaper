\subsection{Lookup Argument}

Lookup Argument is used to prove that the elements of the two sets are of the same type, while the size can differ, therefore, for two sets $A$ and $S$, if we want to prove that they contain the same elements, first we need to remove duplicates and then we obtain two new sets $A'$ and $S'$. By performing LDT on them, we obtain two polynomials $A'(X)$ and $S'(X)$. According to the Schwartz-Zippel Lemma, we can know that the following constraints stand if and only if the set $A'$ contain the same elements as $S'$.
\begin{align*}
    Z(\omega X)(A'(X)+\beta)(S'(X)+\gamma) - Z(X)(A(X)+\beta)(S(X)+\gamma) = 0, \\
    (1-Z(X))l_0(X) = 0.
\end{align*}

For further understanding of the principle, refer to Lookup Argument \cite{website:lookup-argument} and Plookup \cite{cryptoeprint:2020/315}.
