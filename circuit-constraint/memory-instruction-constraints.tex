\subsection{Memory Instruction Constraints}

\subsubsection{MLOAD}

\emph{Instruction:} \verb|MLOAD r_dst, r_src|

The VM logic is
\[ \texttt{[r\_dst] = mem[[r\_src]]} \]

The constraints are
\begin{align*}
    & \texttt{[r\_src}_{i+1}\texttt{]} - \texttt{[r\_src}_i\texttt{]} = 0, \\
    & \texttt{[r\_dst}_{i+1}\texttt{]} - \texttt{mem[[r\_src}_i\texttt{]]} = 0.
\end{align*}

The value of other registers remain unchanged.

\subsubsection{MSTORE}

\emph{Instruction:} \verb|MSTORE r_dst, r_src|

The VM logic is
\[ \texttt{mem[[r\_dst]] = [r\_src]} \]

The constraints are
\begin{align*}
    & \texttt{[r\_src}_{i+1}\texttt{]} - \texttt{[r\_src}_i\texttt{]} = 0, \\
    & \texttt{mem[[r\_dst}_{i+1}\texttt{]]} - \texttt{[r\_src}_i\texttt{]} = 0.
\end{align*}

The value of other registers remain unchanged.

\subsubsection{SLOAD}

\emph{Instruction:} \verb|SLOAD r_dst, r_src|

The VM logic is
\[ \texttt{[r\_dst] = store[[r\_src]]} \]

The constraints are
\begin{align*}
    & \texttt{[r\_src}_{i+1}\texttt{]} = \texttt{[r\_src}_i\texttt{]}, \\
    & \texttt{[r\_dst}_{i+1}\texttt{]} = \texttt{store[[r\_src}_i\texttt{]]}.
\end{align*}

The value of other registers remain unchanged.

\subsubsection{SSTORE}

\emph{Instruction:} \verb|SSTORE r_dst, r_src|

The VM logic is
\[ \texttt{store[[r\_dst]] = [r\_src]} \]

The constraints are
\begin{align*}
    & \texttt{[r\_src}_{i+1}\texttt{]} - \texttt{[r\_src}_i\texttt{]} = 0, \\
    & \texttt{store[[r\_dst}_{i+1}\texttt{]]} - \texttt{[r\_src}_i\texttt{]} = 0.
\end{align*}

The value of other registers remain unchanged.

\subsubsection{Memory Constraint}

The trace of memory is shown in Table \ref{table:memory_trace_columns}.
\begin{table}[!ht]
    \centering \resizebox{0.6\textwidth}{!}{
    \begin{tabular}{|l|l|l|l|l|l|l|l|l|}
        \hline
        ctx & type & addr & clk & rw & v & d0 & d1 & t \\ \hline
    \end{tabular}}
    \caption{Memory trace table}
    \label{table:memory_trace_columns}
\end{table}

In order to handle different contexts of contracts, we need to add context fields. For example, \verb|ctx| of contract address 5 and contract address 6 are 5 and 6 respectively. The meaning of each column of the table is as follows:
\begin{itemize}
\item \verb|ctx|: Context ID. Values in this column must increase monotonically but there can be gaps between two consecutive values of up to $2^{256}$. Two consecutive values can be the same. Abbreviated to \verb|c| in subsequent constraint expressions.
\item \verb|type|: Memory type, such as memory, storage. Currently we define 0 for memory and 1 for storage. Abbreviated to \verb|p| in subsequent constraint expressions.
\item \verb|addr|: Memory/storage address. Values in this column must increase monotonically for a given context but there can be gaps between two consecutive values of up to $2^{256}$. Two consecutive values can be the same. Abbreviated to \verb|a| in subsequent constraint expressions.
\item \verb|clk|: Clock cycle at which the memory operation happened. Values in this column must increase monotonically for a given context and memory address but there can be gaps between two consecutive values of up to $2^{256}$.
\item \verb|rw|: Read/write operation flag, 0 for read and 1 for write.
\item \verb|v|: Field element stored at a given context/address/clock cycle after memory/storage operation.
\item \verb|d0| and \verb|d1|: Lower and upper 128 bits of the delta between two consecutive context IDs, addresses, or clock cycles.
\item \verb|t|: According to the different meanings of the state changes of \verb|ctx|, \verb|adddr| and \verb|clk|, the priorities are as follows:
    \begin{enumerate}
        \item If \verb|ctx| changes, \verb|d0| represents the lower 128 bits of the difference of \verb|ctx|, and \verb|t| represents the inverse of the difference of \verb|ctx|.
        \item If \verb|ctx| remains unchanged but type changes, \verb|d0| represents the lower 128 bits of the difference of \verb|type|, and \verb|t| represents the inverse of the difference of type.
        \item If \verb|ctx| and \verb|type| remain unchanged, but \verb|addr| changes, \verb|d0| represents the lower 128 bits of the difference of address, and \verb|t| represents the inverse of the difference of address.
        \item If \verb|ctx|, \verb|type| and \verb|addr| remain unchanged, but \verb|clk| changes, \verb|d0| represents the lower 128 bits of the difference of \verb|clk| minus 1, and \verb|t| represents the inverse of the difference minus 1 of \verb|clk|, namely $(\texttt{clk}_{i+1}-\texttt{clk}_i-1)^{-1}$.
    \end{enumerate}
\end{itemize}

First we define $n_0$, $n_1$ and $n_2$
\begin{align*}
    n_0 &= (c_{i+1} - c_i) \cdot t_{i+1}, \\
    n_1 &= (p_{i+1} - p_i) \cdot t_{i+1}, \\
    n_2 &= (a_{i+1} - a_i) \cdot t_{i+1}.
\end{align*}

Constrain the change of \verb|ctx| to be consistent with priority 1, and then get the following constraints 1 and 2:
\begin{align*}
    n_0^2 - n_0 &= 0, \\
    (1-n_0)(c_{i+1}-c_i) &= 0.
\end{align*}

Constrain the change of \verb|type| to be consistent with priority 2, and then get the following constraints 3 and 4:
\begin{align*}
    (1-n_0)(n_1^2-n_1) &= 0, \\
    (1-n_0)(1-n_1)(p_{i+1}-p_i) &= 0.
\end{align*}

Constrain the change of \verb|addr| to be consistent with priority 3, and then get the following constraints 5 and 6:
\begin{align*}
    (1-n_0)(1-n_1)(n_2^2-n_2) &= 0, \\
    (1-n_0)(1-n_1)(1-n_2)(a_{i+1}-a_i) &= 0.
\end{align*}

Satisfy the constraint of priority 1 $n_0=1$ and then get the following constraint 7:
\begin{equation*}
    n_0 \left( (c_{i+1}-c_i) - (2^{128} \cdot \texttt{d1}_{i+1} + \texttt{d0}_{i+1}) \right) = 0.
\end{equation*}

Satisfy the constraint of priority 2 $n_0=0, n_1=1$ and then get the following constraint 8:
\begin{equation*}
    (1-n_0)n_1 \left( (p_{i+1}-p_i) - (2^{128} \cdot \texttt{d1}_{i+1} + \texttt{d0}_{i+1}) \right) = 0.
\end{equation*}

Satisfy the constraint of priority 3 $n_0=0, n_1=0, n_2=1$ and then get the following constraint 9:
\begin{equation*}
    (1-n_0)(1-n_1)n_2 \left( (a_{i+1}-a_i) - (2^{128} \cdot \texttt{d1}_{i+1} + \texttt{d0}_{i+1}) \right) = 0.
\end{equation*}

Satisfy the constraint of priority 4 $n_0=0, n_1=0, n_2=0$ and then get the following constraint 10:
\begin{equation*}
    (1-n_0)(1-n_1)(1-n_2) \left( (\texttt{clk}_{i+1}-\texttt{clk}_i-1) - (2^{128} \cdot \texttt{d1}_{i+1} + \texttt{d0}_{i+1}) \right) = 0.
\end{equation*}

Changes of \verb|ctx| or \verb|type| or \verb|addr| will cause the memory value under the new \verb|clk| to be initialized to 0, resulting in constraint 11:
\begin{equation*}
    \big( n_0+(1-n_0)(n_1+(1-n_1)n_2) \big) v_i = 0.
\end{equation*}

Changes of \verb|ctx|, \verb|type| or \verb|addr| will cause the \verb|rw| under the new \verb|clk| to be initialized to 1, resulting in constraint 12:
\begin{equation*}
    \big( n_0+(1-n_0)(n_1+(1-n_1)n_2) \big) (1-\texttt{rw}) = 0.
\end{equation*}

When \verb|ctx|, \verb|type| and \verb|addr| remain unchanged and \verb|rw| is the \verb|read| type, the old value of the new \verb|clk| cycle of memory is equal to the new value of the previous \verb|clk| cycle. The combination $(\texttt{ctx},\texttt{type},\texttt{addr},\texttt{clk})$ is unique, then obtain constraint 13:
\begin{equation*}
    (1-n_0)(1-n_1)(1-n_2) (1-\texttt{rw}) (v_{i+1}-v_i) = 0.
\end{equation*}
