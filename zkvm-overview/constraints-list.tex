\subsection{Constraint List}

Because of the modular design, each module handles a specific function, resulting in specific constraints for each module, described in detail in later sections. Furthermore, we added context constraints, primarily used to constrain the change of context after the VM executes each instruction, this constraint applies to all instructions, hence applied to the main trace table.

According to the instruction structure we defined, it occupies 31 valid bits, enabling us to state the following definitions:

Arithmetic operations:
\begin{align*}
    f_\text{ADD} = f_{0},\ f_\text{SUB} = f_{1},\ f_\text{MUL} = f_{2},\ f_\text{DIV} = f_{3},
\end{align*}

Control flows:
\begin{align*}
    f_\text{JMP} = f_{4},\ f_\text{CJMP} = f_{5},\ f_\text{CALL} = f_{6},\ f_\text{RET} = f_{7},\ f_\text{END} = f_{8},
\end{align*}

Moves:
\begin{align*}
    f_\text{MOV} = f_{9},\ f_\text{MOVI} = f_{10},
\end{align*}

Memory and storage operations:
\begin{align*}
    f_\text{MLOAD} = f_{11},\ f_\text{MSTORE} = f_{12},\ f_\text{SLOAD} = f_{13},\ f_\text{SSTORE} = f_{14},
\end{align*}

External inputs:
\begin{align*}
    f_\text{INPUT} = f_{15},\ f_\text{EX\_INPUT} = f_{23},
\end{align*}

\verb|pc|:
\begin{align*}
    f_\text{PC\_CALL} = f_{16},\ f_\text{PC\_RET} = f_{17},\ f_\text{PC\_JMP} = f_{18},\ f_\text{PC\_CJMP} = f_{19},\ f_\text{PC\_END} = f_{20},
\end{align*}

\verb|fp| and \verb|const|:
\begin{align*}
    f_\text{FP\_CALL} = f_{21},\ f_\text{FP\_RET} = f_{22},\ f_\text{CONST} = f_{24}
\end{align*}

Inputs:
\begin{align*}
    f_\text{INPUT\_R0} = f_{25},\ f_\text{INPUT\_R1} = f_{26},\ f_\text{INPUT\_R2} = f_{27},
\end{align*}

Outputs:
\begin{align*}
    f_\text{UPDATE\_R0} = f_{28},\ f_\text{UPDATE\_R1} = f_{29},\ f_\text{UPDATE\_R2} = f_{30}.
\end{align*}

\subsubsection{Inst-decode Check}

Upon execution, the VM decodes the instruction, then performs the corresponding operation according to the parsed flag. Therefore, we need to verify that the instruction executed by the VM is valid, i.e.\ the value of the instruction encoding is consistent with the corresponding flag. For simplicity, we will not allocate a separate column (the length is $N$) for all flags, however, we will use a virtual column to represent (the length is $32N$). Therefore we define $\widetilde f_i = \sum_{j=i}^{30} 2^{j-i} f_j$, so $\widetilde f_0 = \sum_{j=0}^{30} 2^{j-0} f_j$, which represents the value of the first 31 bits of the instruction and $\widetilde f_{31} = 0$. In order to obtain the value of $f_i$ from $\{\widetilde f_i \}_{i=0}^{31}$, we utilize the equation
\begin{align*}
    \widetilde f_i - 2\widetilde f_{i+1} &= \sum_{j=i}^{30} 2^{j-i} f_j - 2 \sum_{j=i+1}^{30} 2^{j-i-1} f_j \\
    &= \sum_{j=i}^{30} 2^{j-i} f_j - \sum_{j=i+1}^{30} 2^{j-i} f_j. \\
    &= f_i
\end{align*}

The corresponding constraints are as follows:

Instruction:
\[ \text{instruction} = \widetilde f_0, \]

Bit:
\[ (\widetilde f_i - 2\widetilde f_{i+1} )(\widetilde f_i - 2\widetilde f_{i+1} - 1) = 0, \qquad \forall i \in [0,31), \]

Last value is zero:
\[ \widetilde f_{31} = 0. \]

\subsubsection{Context Update}

If the instruction is \verb|JMP| or \verb|CJMP|, \verb|pc| is updated to the specified location \verb|dst_pc|, and register \verb|fp| stores the address of RPR.

If the instruction is \verb|CALL|, \verb|pc| is updated to the value in \verb|r0|, and the value of the location where \verb|fp| points to the RPR is set to \verb|pc+instruction_size|. \verb|fp+1| points to an empty memory area.

If the instruction is \verb|RET|, the function call ends, and \verb|pc| is updated to the value saved by the RPR pointed to by \verb|fp|, and \verb|fp| is updated to the value of \verb|fp-1| in RPR.

\begin{multline*}
    f_\text{pc}(x\omega) - f_\text{PC\_*JMP}(f_\text{dst\_{pc}}(x) + f_\text{PC\_CALL}(f_\text{dst\_{pc}}(x)) +f_\text{PC\_RET}(f_\text{fp}(x)) + \\
    (1 - f_\text{PC\_*JMP} - f_\text{PC\_CALL} - f_\text{PC\_RET})(f_\text{pc}(x) + \text{instruction\_size})) = 0,
\end{multline*}

\[ f_\text{FP\_CALL}(f_\text{fp}(x\omega) - (f_\text{pc}(x) + \text{instruction\_size})) = 0. \]

\subsubsection{Register Check}

We have three general-purpose registers. Most of the instructions are based on registers. During execution of instructions, some register states may change while others stay the same, based on which the instruction is being executed. We define flags to indicate the register read/write state corresponding to the current instruction. We can constrain these general-purpose registers through flags:
\begin{align*}
    (1 - f_\text{r0\_output})(f_\text{r0}(x\omega) - f_\text{r0}(x)) &= 0, \\
    (1 - f_\text{r1\_output})(f_\text{r1}(x\omega) - f_\text{r1}(x)) &= 0, \\
    (1 - f_\text{r2\_output})(f_\text{r2}(x\omega) - f_\text{r2}(x)) &= 0.
\end{align*}

The above constraints ensures that if the current instruction did not update the corresponding register, the value inside the register should stay the same, if the value of a certain register changed, we will ensure the validity of the value change on the corresponding sub trace table. For example, when executing the instruction \verb|ADD r0,r1,r2|, we will verify that the value inside register \verb|r1| and register \verb|r2| remains the same, then check \verb|r0=r1+r2| on the arithmetic sub trace table.

\subsubsection{Execute Correct Program}

The constraints described in previous sections guarantee the correctness of all the instructions executed by the VM, but there is no guarantee that the programs corresponding to these instructions are correct, i.e.\ corresponding to our original program. Therefore, we also need to verify that we have executed the correct program. Example given, we have the following program and store it in ROM:

\begin{table}[!ht]
    \centering
    \begin{tabular}{|c|c|c|}
    \hline
        \emph{pc} & \emph{Program Bytecodes} & \emph{program\_encode} \\ \hline
        0 & READ R0 & 0000000000000001000000001000010000 \\
        1 & MOV R1 3 & 0000000001000000000000000100001001 \\
        2 & ADD R0, R1, R2 & 0001000000000000000000000011000110 \\
        3 & END & 0000100000000000100000000000011111 \\ \hline
    \end{tabular}
    \caption{Simple VM example}
    \label{table:simple-vm-example}
\end{table}

Given an input, we get the execution trace table as shown in Table \ref{table:execution-trace-encoding}.

\begin{table}[!ht]
    \resizebox{\textwidth}{!}{
    \begin{tabular}{|c|c|c|c|c|c|c|c|c|c|c|c|c|c|}
    \hline
    clk & program\_encode & arith\_op & flow\_op & move\_op & ram\_op & input\_op & pc\_update & fp\_update & ext\_input & const\_input & r\_input & r\_output & pc \\ \hline
    0 & 0000000000000001000000001000010000 & 0000 & 00000 & 00 & 0000 & 1 & 00000 & 000 & 1 & 0 & 000 & 100 & 0 \\
    1 & 0000000001000000000000000100001001 & 0000 & 00000 & 10 & 0000 & 0 & 00000 & 000 & 0 & 1 & 000 & 010 & 1 \\
    2 & 0001000000000000000000000011000110 & 0001 & 00000 & 00 & 0000 & 0 & 00000 & 000 & 0 & 0 & 110 & 001 & 2 \\
    3 & 0000100000000000100000000000011111 & 0000 & 10000 & 00 & 0000 & 0 & 10000 & 000 & 0 & 0 & 000 & 111 & 3 \\
    4 & $\cdots$ & $\cdots$ & $\cdots$ & $\cdots$ & $\cdots$ & $\cdots$ & $\cdots$ & $\cdots$ & $\cdots$ & $\cdots$ & $\cdots$ & $\cdots$ & $\cdots$ \\ \hline
    \end{tabular}
    }
    \caption{Execution trace table example}
    \label{table:execution-trace-encoding}
\end{table}

For simplification, the maximum \verb|pc| is set to 4, we will utilize Lookup Argument and program encoding form shown below to ensure the execution of the correct program:
\[ \text{trace\_program\_encode} = 2^2 \cdot \text{instruction\_encode} + \text{pc} \]
\[ \{\text{trace\_program\_encode}\} \subset \{\text{ROM\_program\_encode}\} \]

\emph{Note:}
\begin{enumerate}
    \item The correctness of the instruction is verified by Inst-decode check;
    \item The immediate value after the instruction does not need to be encoded, but \verb|pc| needs to be encoded into the instruction, and the correctness of the immediate value after the instruction is guaranteed by the equality circuit.
\end{enumerate}
